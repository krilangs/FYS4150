\documentclass[12pt,a4paper,english]{article}
\usepackage{times}
\usepackage[utf8]{inputenc}
\usepackage{babel,textcomp}
\usepackage{mathpazo}
\usepackage{mathtools}
\usepackage{amsmath,amssymb}
\usepackage{ dsfont }
\usepackage{listings}
\usepackage{graphicx}
\usepackage{float}
\usepackage{epsfig,floatflt}
\usepackage{subfig} 
\usepackage[colorlinks]{hyperref}
\usepackage[hyphenbreaks]{breakurl}
\usepackage[usenames,dvipsnames,svgnames,table]{xcolor}
\usepackage{textcomp}
\definecolor{listinggray}{gray}{0.9}
\definecolor{lbcolor}{rgb}{0.9,0.9,0.9}
\lstset{backgroundcolor=\color{lbcolor},tabsize=4,rulecolor=,language=python,basicstyle=\scriptsize,upquote=true,aboveskip={1.5\baselineskip},columns=fixed,numbers=none,showstringspaces=false,extendedchars=true,breaklines=true,
prebreak=\raisebox{0ex}[0ex][0ex]{\ensuremath{\hookleftarrow}},frame=single,showtabs=false,showspaces=false,showstringspaces=false,identifierstyle=\ttfamily,keywordstyle=\color[rgb]{0,0,1},commentstyle=\color[rgb]{0.133,0.545,0.133},stringstyle=\color[rgb]{0.627,0.126,0.941},literate={å}{{\r a}}1 {Å}{{\r A}}1 {ø}{{\o}}1}

% Use for references
%\usepackage[sort&compress,square,comma,numbers]{natbib}
%\DeclareRobustCommand{\citeext}[1]{\citeauthor{#1}~\cite{#1}}

% Fix spacing in tables and figures
%\usepackage[belowskip=-8pt,aboveskip=5pt]{caption}
%\setlength{\intextsep}{10pt plus 2pt minus 2pt}

% Change the page layout
%\usepackage[showframe]{geometry}
\usepackage{layout}
\setlength{\hoffset}{-0.6in}  % Length left
%\setlength{\voffset}{-1.1in}  % Length on top
\setlength{\textwidth}{480pt}  % Width /597pt
%\setlength{\textheight}{720pt}  % Height /845pt
%\setlength{\footskip}{25pt}

\newcommand{\VEV}[1]{\langle#1\rangle}
\title{FYS4150 - Project 2}
\date{}
\author{ Kristoffer Langstad\\ \textit{krilangs@uio.no}}

\begin{document}%layout
\maketitle
\begin{abstract}
In this project we want to solve eigenvalue problems of a tridiagonal Toeplitz matrix using Jacobi's method. We will use unit tests to check our results with other solvers like Python's numpy. We will solve different physical problems by discretization and scaling to make them into similar differential equations that we solve numerically. The first problem is a classical wave function problem in one dimension as a two-point boundary value problem of a buckling beam. This eigenvalue problem has analytical solutions which we will use as a unit test against our numerical algorithm. We also check that the eigenvectors retain their orthogonality after the transformations. Both these tests passed and gave similar eigenvalues within a small tolerance of 1e-8. Next we look at a quantum mechanical problem for an electron in a 3-dimensional harmonic oscillator (quantum dots). This case also have analytical eigenvalues which we test against. For our numerical simulation we get eigenvalues $\lambda=[3.0025, 7.0026, 10.9988]$ for the first three eigenvalues, with 555'255 transformations for a $800\times800$ matrix and a $\rho_{max}=50$. For the last case we expand for two electrons which interact via a repulsive Coulomb interaction. Here we test for different frequency strengths $\omega_r=[0.01, 0.1, 1, 5]$ for the ground state. For these frequencies there are analytical solutions that we compare with...
\end{abstract}

\section{Introduction}
\end{document}
