\documentclass[12pt,a4paper,english]{article}
\usepackage{times}
\usepackage[utf8]{inputenc}
\usepackage{babel,textcomp}
\usepackage{mathpazo}
\usepackage{mathtools}
\usepackage{amsmath,amssymb}
\usepackage{ dsfont }
\usepackage{listings}
\usepackage{graphicx}
\usepackage{ mathrsfs }
\usepackage{float}
\usepackage{subfig} 
\usepackage[colorlinks]{hyperref}
\usepackage[usenames,dvipsnames,svgnames,table]{xcolor}
\usepackage{textcomp}
\definecolor{listinggray}{gray}{0.9}
\definecolor{lbcolor}{rgb}{0.9,0.9,0.9}
\lstset{backgroundcolor=\color{lbcolor},tabsize=4,rulecolor=,language=python,basicstyle=\scriptsize,upquote=true,aboveskip={1.5\baselineskip},columns=fixed,numbers=left,showstringspaces=false,extendedchars=true,breaklines=true,
prebreak=\raisebox{0ex}[0ex][0ex]{\ensuremath{\hookleftarrow}},frame=single,showtabs=false,showspaces=false,showstringspaces=false,identifierstyle=\ttfamily,keywordstyle=\color[rgb]{0,0,1},commentstyle=\color[rgb]{0.133,0.545,0.133},stringstyle=\color[rgb]{0.627,0.126,0.941},literate={å}{{\r a}}1 {Å}{{\r A}}1 {ø}{{\o}}1}

% Use for references
\usepackage[square,comma,numbers]{natbib}
%\DeclareRobustCommand{\citeext}[1]{\citeauthor{#1}~\cite{#1}}

% Fix spacing in tables and figures
%\usepackage[belowskip=0pt,aboveskip=5pt]{caption}
%\setlength{\intextsep}{10pt plus 2pt minus 2pt}

% Change the page layout
%\usepackage[showframe]{geometry}
\usepackage{layout}
\setlength{\hoffset}{-0.5in}  % Length left
%\setlength{\voffset}{-0.8in}  % Length on top
\setlength{\textwidth}{470pt}  % Width /597pt
%\setlength{\textheight}{670pt}  % Height /845pt
%\setlength{\footskip}{25pt}

\newcommand{\VEV}[1]{\langle#1\rangle}
\title{FYS4150 - Project 5\\
The Solar System}
\date{}
\author{ Kristoffer Langstad\\ \textit{krilangs@uio.no}}

\begin{document}
\maketitle
\begin{abstract}
	In this project we want to model the solar system using ordinary differential equations and object orientation numerically. To do this we will use the so-called velocity Verlet algorithm. We start of with only the Earth and the Sun in the system, which we will solve with both Euler's forward algorithm and the velocity Verlet algorithm. For this simple two-body system we write the code to be object oriented and scaled with the Sun mass as $M_\odot=1$, and test the stability of the two algorithms. The Verlet algorithm is found to be superior in accuracy since the algorithm conserves the energy of the system, while the forward Euler algorithm does not. The Euler algorithm is faster, but the difference is so small that the Verlet algorithm is the best to use. Then we find properties like the escape velocity of Earth needed to escape the orbit of the Sun. This is found analytically to be $\sqrt{8}\pi\frac{\text{AU}}{\text{yr}}\approx8.88577\frac{\text{AU}}{\text{yr}}$, and numerically as $8.88250\frac{\text{AU}}{\text{yr}}$ with the velocity Verlet algorithm. By changing the gravitational force factor $\frac{1}{r^2}\rightarrow\frac{1}{r^{\beta}}$, we see that as $\beta\rightarrow3$ the escape velocity of the Earth escaping the Sun's orbit is decreasing more and more. So by changing the force factor, we change how strongly the orbit of the Earth is affected by the Sun. Then the system is altered to a three-body problem by introducing Jupiter to the system. First we keep the Sun as the center of mass of the system and study once again the stability of the Verlet algorithm. For 1x and 10x times the mass of Jupiter, the system is more or less stable, but when increasing the Jupiter mass by 1000 the system becomes chaotic since the Jupiter mass is around the same as the Sun mass. Now by setting the center of mass of the system as the origin of the Solar system, we add the rest of the planets (including Pluto) in the Solar system. Up till now we have used Newtonian physics, so we add a general relativistic correction to the gravitation force. This is used on the Sun-Mercury system to analyze the perihelion precession of Mercury. This observed value is found to be 43 arc seconds per century, while our numerical value is found to be 43.1281 arc seconds per century.
\end{abstract}

\section{Introduction}
\label{sect:Introduction}
In many cases, like the Solar system, we have many coupled ordinary differential equations which in many ways are similar to each other. Often the only differences are constants and variables that differ them from each other. So to reduce the amount of code and to make are lives much easier for solving these equations, we can use object orientation. This makes it easy extend and to add more elements without having to add a lot of new code. The code is then rerun several times instead. This is done in this project by implementation of classes to make the code more reusable.

In this project we will solve the Solar system with ordinary differential equations with an object oriented code. The algorithms we will use are the forward Euler algorithm and the velocity Verlet algorithm. In the first part we study the two-body Sun-Earth only system to test the stability of the algorithms. Here we test the stability with different time steps and test for energy conservation in the system for the two algorithms. Then we calculate system properties like the escape velocity of Earth, and test what happens when we change the gravitational force between the objects. With the object oriented code fully set up, we can easily add other object to the system like Jupiter. The system is now a three-body system. Here we study what happens to the system when we increase the mass of Jupiter. Then we extend to include all the planets in the Solar system (including Pluto), which is easily done by adding masses and initial conditions of the planets and running the classes more times. The initial conditions of the Sun and the planets are found at a site by NASA \cite{horizon}, which contains data like the masses, initial positions and initial velocities of the objects up to three dimensions. Lastly we study the perihelion precession of Mercury by introducing a general relativistic correction to our Newtonian gravitational force, for a system containing only the Sun and Mercury.

In the methods section we look at the theory and implementation of the algorithms and of the physics connected to our problem necessary for us to solve the ordinary differential equations and properties for the planets. This includes topics like unit scaling, derivation of the numerics, the equations for energy and angular momentum and the escape velocity, and perihelion precession of Mercury. In the results section we will look at and analyze the results we get from the methods section. Here we will discuss the results and compare with analytically found results for verification of our numerics. Last in the conclusion section, we give an overall conclusion of the results of the project we have gotten, and possible future tasks that can be done.

\section{Methods}
\label{sect:Method}
\section{Results}
\label{sect:Results}
\section{Conclusion}
\label{sect:Conclusion}

\appendix
\section{Appendix}
\label{sect:appendix}
Link to GitHub repository:\\
\url{https://github.com/krilangs/FYS4150/tree/master/Project5}

\bibliographystyle{plainnat}
\bibliography{myrefs}
\end{document}
